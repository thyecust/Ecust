% This LaTeX was auto-generated from MATLAB code.
% To make changes, update the MATLAB code and export to LaTeX again.

\documentclass{article}

\usepackage[utf8]{inputenc}
\usepackage[T1]{fontenc}
\usepackage{lmodern}
\usepackage{graphicx}
\usepackage{color}
\usepackage{listings}
\usepackage{hyperref}
\usepackage{amsmath}
\usepackage{amsfonts}
\usepackage{epstopdf}
\usepackage{matlab}
\usepackage{ctex}
\usepackage{geometry}
\geometry{left=3.0cm,right=2.0cm,top=2.5cm,bottom=2.5cm}

\sloppy
\epstopdfsetup{outdir=./}
\graphicspath{ {./mln01_images/} }

\matlabmultipletitles
\matlabhastoc

\begin{document}
\label{T_493DB8AF}
\matlabtitle{MATLAB01 MATLAB概述}

\begin{par}
\begin{flushleft}
本课程讲授MATLAB语言基础入门知识,重点介绍MATLAB初等数学运算、MATLAB的数据可视化及如何用MATLAB语言编写整洁、高效、规范的程序。
\end{flushleft}
\end{par}

\begin{par}
\begin{flushleft}
课程大纲:
\end{flushleft}
\end{par}

\begin{itemize}
\setlength{\itemsep}{-1ex}
   \item{\begin{flushleft} Chapter1 MATLAB概述及入门 \end{flushleft}}
   \item{\begin{flushleft} Chapter2 初等数学运算 \end{flushleft}}
   \item{\begin{flushleft} Chapter3 符号运算 \end{flushleft}}
   \item{\begin{flushleft} Chapter4 数据和函数的可视化 \end{flushleft}}
   \item{\begin{flushleft} Chapter5 MATLAB编程 \end{flushleft}}
   \item{\begin{flushleft} Chapter6 数值计算 \end{flushleft}}
\end{itemize}

\matlabtableofcontents{目录}

\label{T_F4DC9618}
\matlabtitle{MATLAB概述}
\label{H_8F21B05B}
\matlabheading{MATLAB发展历史}

\begin{par}
\begin{flushleft}
在20世纪70年代中期,美国Cleve Moler博士和 其同事在美国国家科学基金的资助下开发了调 用EISPACK和LINPACK的FORTRAN子程序 库。EISPACK是特征值求解的FORTRAN程序 库, LINPACK是解线性方程的程序。
\end{flushleft}
\end{par}

\begin{par}
\begin{flushleft}
20世纪70年代后期, Cleve Moler编写接口程序 : MATLAB,即矩阵(matrix)和实验室(laboratory)前3个字母的组合,是“矩阵实验室”的缩写,它是一种以矩阵运算为基础的交互式程序语言。
\end{flushleft}
\end{par}

\begin{par}
\begin{flushleft}
1983年,Cleve Moler和Jack Little用C语言开发了MATLAB第二代专业版。
\end{flushleft}
\end{par}

\begin{par}
\begin{flushleft}
1984年,两人成立了Mathworks公司, 正式把MATLAB推向市场。
\end{flushleft}
\end{par}
\label{H_0AD881BF}
\matlabheading{MATLAB特点}

\begin{par}
\begin{flushleft}
例1,解方程x\textasciicircum{}2-5x+6=0
\end{flushleft}
\end{par}

\begin{matlabcode}
p = [1 -5 6]
\end{matlabcode}
\begin{matlaboutput}
p = 1x3    
     1    -5     6

\end{matlaboutput}
\begin{matlabcode}
roots(p)
\end{matlabcode}
\begin{matlaboutput}
ans = 2x1    
    3.0000
    2.0000

\end{matlaboutput}

\begin{par}
\begin{flushleft}
例2,绘制Schaffer函数
\end{flushleft}
\end{par}

\begin{matlabcode}
clc
clear
close all
x = [-10:0.05:10];
y = x;
[X,Y] = meshgrid(x,y);
[row,col]=size(X);
Z = 0.5 + (sin(sqrt(X.^2 + Y.^2))^2 - 0.5)./(1 + 0.001*(X.^2 + Y.^2)).^2;
surf(X,Y,Z);
shading interp
\end{matlabcode}
\begin{center}
\includegraphics[width=\maxwidth{56.196688409433015em}]{figure_0}
\end{center}

\label{T_8F00169D}
\matlabtitle{MATLAB入门知识}
\label{H_22B2CBB9}
\matlabheading{MATLAB中的几个简单函数}

\begin{par}
\begin{flushleft}
sin(),正弦函数,以弧度为单位
\end{flushleft}
\end{par}

\begin{par}
\begin{flushleft}
sqrt(),开平方函数
\end{flushleft}
\end{par}

\begin{par}
\begin{flushleft}
exp(),自然指数函数
\end{flushleft}
\end{par}
\label{H_35EE2C1E}
\matlabheading{MATLAB中的变量}

\begin{par}
\begin{flushleft}
变量不需要声明,不需要指定类型,使用时必须先赋值。
\end{flushleft}
\end{par}

\begin{par}
\begin{flushleft}
变量名由字母、数字和下划线组成,第一个字母必须是英文字母;变量名中的英文字母大小写是有区别的;变量名只有前63位为MATLAB所认可,其余将被忽略;变量名应不和MATLAB关键字相同。
\end{flushleft}
\end{par}

\begin{par}
\begin{flushleft}
特殊变量:
\end{flushleft}
\end{par}

\begin{par}
\begin{flushleft}
\textbf{pi }=3.14159...
\end{flushleft}
\end{par}

\begin{par}
\begin{flushleft}
\textbf{ans} 如果未定义变量名,用于计算结果存储的默认临时变量名
\end{flushleft}
\end{par}

\begin{par}
\begin{flushleft}
\textbf{inf或Inf} 无穷大∞值
\end{flushleft}
\end{par}

\begin{par}
\begin{flushleft}
\textbf{eps} MATLAB定义的正的极小值
\end{flushleft}
\end{par}

\begin{par}
\begin{flushleft}
\textbf{NaN或nan} 非数或不定数(如: 0/0)
\end{flushleft}
\end{par}

\begin{par}
\begin{flushleft}
\textbf{i或j} 虚数单位,sqrt(-1)
\end{flushleft}
\end{par}
\label{H_42732752}
\matlabheading{MATLAB中的几个窗口命令}

\begin{par}
\begin{flushleft}
clear 删除工作空间中的变量
\end{flushleft}
\end{par}

\begin{par}
\begin{flushleft}
whos 展示工作空间中的变量
\end{flushleft}
\end{par}

\begin{par}
\begin{flushleft}
cd 显示或改变工作目录 
\end{flushleft}
\end{par}

\begin{par}
\begin{flushleft}
dir 显示目录文件 
\end{flushleft}
\end{par}

\begin{par}
\begin{flushleft}
type 显示文件内容
\end{flushleft}
\end{par}

\begin{par}
\begin{flushleft}
clf 清除图形窗口 
\end{flushleft}
\end{par}

\begin{par}
\begin{flushleft}
pack 收集碎片,扩大内存空间
\end{flushleft}
\end{par}

\begin{par}
\begin{flushleft}
clc 清除命令窗口内容
\end{flushleft}
\end{par}

\begin{par}
\begin{flushleft}
echo 命令窗口信息显示开关 
\end{flushleft}
\end{par}

\begin{par}
\begin{flushleft}
hold 图形保持开关 
\end{flushleft}
\end{par}

\begin{par}
\begin{flushleft}
disp 显示变量或文字内容
\end{flushleft}
\end{par}

\begin{par}
\begin{flushleft}
path 显示搜索目录 
\end{flushleft}
\end{par}

\begin{par}
\begin{flushleft}
save 保存内存变量到指定文件
\end{flushleft}
\end{par}

\begin{par}
\begin{flushleft}
load 加载指定文件变量 
\end{flushleft}
\end{par}

\begin{par}
\begin{flushleft}
diary 日志文件命令 
\end{flushleft}
\end{par}

\begin{par}
\begin{flushleft}
quit 退出MATLAB 
\end{flushleft}
\end{par}

\begin{par}
\begin{flushleft}
! 调用DOS命令
\end{flushleft}
\end{par}
\label{H_092A429F}
\matlabheading{MATLAB操作页面}

\begin{par}
\begin{flushleft}
\textbf{命令行窗口,}[Enter]提交命令执行,"\textgreater{}\textgreater{}"是默认的命令提示符,不含赋值号的表达式赋给默认变量"ans",需要多行书写时使用三个或以上黑点表示下一行是上一行的继续。
\end{flushleft}
\end{par}

\begin{par}
\begin{flushleft}
\textbf{命令历史窗口},显示在命令窗口中输入的所有命令,想再次执行时只需双击。
\end{flushleft}
\end{par}

\begin{par}
\begin{flushleft}
如:
\end{flushleft}
\end{par}

\begin{par}
\begin{flushleft}
Cut或Copy 删除、复制
\end{flushleft}
\end{par}

\begin{par}
\begin{flushleft}
Evaluate Selection 执行选定的命令行,并在命令窗口中显示运行结果
\end{flushleft}
\end{par}

\begin{par}
\begin{flushleft}
Create Script 打开M文件编辑/调试器, 将选中的历史命令复制到编辑器,创建M文件
\end{flushleft}
\end{par}

\begin{par}
\begin{flushleft}
Create Shortcut 为选中的某条或某段表达式或命令在快捷工具栏中创建快捷按钮
\end{flushleft}
\end{par}

\begin{par}
\begin{flushleft}
Delete Selection 删除所选历史命令
\end{flushleft}
\end{par}

\begin{par}
\begin{flushleft}
Delete to Selection 删除所选历史命令之前的所有历史命令
\end{flushleft}
\end{par}

\begin{par}
\begin{flushleft}
Clear Command History 清除历史命令
\end{flushleft}
\end{par}

\begin{par}
\begin{flushleft}
\textbf{当前目录窗口,}显示当前目录,提供搜索功能。
\end{flushleft}
\end{par}

\begin{par}
\begin{flushleft}
MATLAB的搜索路径:检查是不是一个变量,检查是不是一个内部函数,检查是否是当前目录下的M文件,减势是否是搜索路径中其他目录下的M文件。
\end{flushleft}
\end{par}

\begin{par}
\begin{flushleft}
设置搜索路径:
\end{flushleft}
\end{par}

\begin{itemize}
\setlength{\itemsep}{-1ex}
   \item{\begin{flushleft} path(path,'some path') \end{flushleft}}
   \item{\begin{flushleft} 主页标签-环境选项卡-设置路径 / pathtool命令,通过Add Folder和Add with Subfolder添加到列表,保存。 \end{flushleft}}
\end{itemize}

\begin{par}
\begin{flushleft}
\textbf{工作空间窗口},显示变量。
\end{flushleft}
\end{par}
\label{H_E7D714C7}
\matlabheading{MATLAB的帮助系统}

\begin{par}
\begin{flushleft}
使用窗口查询帮助的方法:
\end{flushleft}
\end{par}

\begin{itemize}
\setlength{\itemsep}{-1ex}
   \item{\begin{flushleft} help命令 \end{flushleft}}
   \item{\begin{flushleft} lookfor (-all)命令 \end{flushleft}}
\end{itemize}

\begin{par}
\begin{flushleft}
使用联机帮助的方法:
\end{flushleft}
\end{par}

\begin{itemize}
\setlength{\itemsep}{-1ex}
   \item{\begin{flushleft} F1 \end{flushleft}}
   \item{\begin{flushleft} 主页标签-资源选项卡-帮助-文档 \end{flushleft}}
   \item{\begin{flushleft} help win, help desk(将要淘汰), doc命令 \end{flushleft}}
\end{itemize}
\label{H_565327AD}
\matlabheading{MATLAB的联机演示系统}

\begin{par}
\begin{flushleft}
打开系统的方式:
\end{flushleft}
\end{par}

\begin{itemize}
\setlength{\itemsep}{-1ex}
   \item{\begin{flushleft} demos命令 \end{flushleft}}
   \item{\begin{flushleft} 主页标签-资源选项卡-帮助-实例 \end{flushleft}}
   \item{\begin{flushleft} MATLAB帮助文档的目录-Examples \end{flushleft}}
\end{itemize}

\end{document}
